\section*{Cel i zakres pracy}
\addcontentsline{toc}{section}{Cel i zakres pracy}

\subsection*{Cel pracy}
\addcontentsline{toc}{subsection}{Cel pracy}
Celem niniejszej pracy jest opracowanie oraz implementacja testów wydajnościowych oraz analiza wyników opracowanych testów. Wraz z testami zostanie przygotowana strona internetowa, której zadaniem jest prezentowanie wyników danego testu w oparciu o React \cite{React} oraz FastAPI \cite{FastAPI}. Dla każdego testu zostanie przygotowany wykres z czasami wykonania dla danej iteracji, ilość pamięci wykorzystanej wraz z zużyciem procesora w trakcie wykonywania testu. Test wydajnościowy dla testu związanego z żądaniami \nomrefeq{HTTP} został rozszerzony o statystyki, które posiada program oha \cite{oha}.

\subsection*{Zakres pracy}
\addcontentsline{toc}{subsection}{Zakres pracy}
Zakres pracy obejmuje zbudowanie testów wydajnościowych w językach JavaScript oraz Typescript dla wybranych środowisk uruchomieniowych JavaScript oraz przedstawienie wyników testów oraz ich analizę. Umożliwiająca analizę wyników testów została zbudowana aplikacja w oparciu o React \cite{React} oraz FastAPI \cite{FastAPI} aplikacja webowa, która przestawia dane w formie wykresu wraz z danymi o zużyciu procesora oraz pamięci \nomrefeq{RAM}.

\subsection*{Układ pracy}
\addcontentsline{toc}{subsection}{Układ pracy}
W tej sekcji znajduje się opis poszczególnych rozdziałów:
\begin{enumerate}
  \item Wstęp - rozdział, który przedstawia problematykę pracy,
  \item Cel i zakres pracy - rozdział, w którym opisany jest cel oraz zakres pracy,
  \item Przegląd środowisk uruchomieniowych - rozdział, w którym opisane są wybrane środowiska uruchmomieniowe
  \item Testy wydajnościowe - rozdział, w którym zostały opisane testy użyte do badań,
  \item Wyniki testów - rozdział, w którym zostały przedstawione wyniki testów,
  \item Dyskusja wyników - rozdział, w którym przedstawiona analiza wyników testów, 
  \item Podsumowanie - rozdział, w którym zostały przedstawione wnioski dotyczące wyników.
\end{enumerate}