\section*{Streszczenie}
\addcontentsline{toc}{section}{Streszczenie}
Praca magisterska pt. "Analiza wydajnościowa środowisk uruchomieniowych JavaScript" koncentruje się na szczegółowym badaniu i porównaniu najważniejszych środowisk uruchomieniowych JavaScript, które są powszechnie wykorzystywane w nowoczesnym programowaniu webowym oraz serwerowym. W szczególności analizowane są Node.js, Deno oraz Bun.

Głównym celem pracy jest zidentyfikowanie różnic, zalet i wad każdego ze środowisk oraz ocena ich przydatności w różnych scenariuszach programistycznych. Praca ma na celu dostarczenie programistom i decydentom technologicznym kompleksowej wiedzy, która pomoże im w wyborze odpowiedniego narzędzia do realizacji swoich projektów.

W pracy zastosowano kombinację metod badawczych, w tym przegląd literatury, eksperymentalne testy wydajności, analizę funkcjonalności oraz ankiety wśród programistów. Przeprowadzone eksperymenty obejmowały pomiar wydajności w różnych scenariuszach (np. obsługa dużej liczby żądań HTTP, operacje wejścia/wyjścia, przetwarzanie danych) oraz ocenę łatwości użycia i dostępności bibliotek.

Node.js - Najbardziej dojrzałe i szeroko stosowane środowisko uruchomieniowe, oferujące bogaty ekosystem bibliotek NPM oraz stabilność w produkcji. Wyróżnia się doskonałą obsługą I/O oraz wsparciem dla długoterminowych projektów.

Deno - Nowoczesne środowisko stworzone przez Ryana Dahla, twórcę Node.js, które kładzie duży nacisk na bezpieczeństwo i zgodność z nowoczesnymi standardami ECMAScript. Deno integruje TypeScript bezpośrednio, co eliminuje konieczność dodatkowej konfiguracji.

Bun - Najnowsze środowisko uruchomieniowe, skoncentrowane na wydajności i prostocie użycia. Charakteryzuje się szybkim czasem startu oraz niskim zużyciem zasobów, co czyni je obiecującym wyborem dla aplikacji wymagających wysokiej wydajności.

Przeprowadzone analizy wskazują, że każde ze środowisk ma swoje unikalne zalety i wady, które czynią je odpowiednimi dla różnych typów projektów.

\bigskip

\textbf{Słowa kluczowe}: JavaScript, Wydajność, Deno, Bun, NodeJS
\newpage

\section*{Abstract}
\addcontentsline{toc}{section}{Abstract}
The thesis entitled. ‘Performance analysis of JavaScript runtime environments’ focuses on a detailed study and comparison of the most important JavaScript runtime environments that are commonly used in modern web and server-side programming. In particular, Node.js, Deno and Bun are analyzed.

The main aim of the work is to identify the differences, advantages and disadvantages of each environment and to evaluate their suitability in different programming scenarios. The work aims to provide developers and technology decision makers with comprehensive knowledge to help them choose the right tool for their projects.

The work used a combination of research methods, including a literature review, experimental performance testing, functionality analysis and developer surveys. The experiments conducted included measuring performance in various scenarios (e.g. handling a large number of HTTP requests, input/output operations, data processing) and assessing the ease of use and accessibility of the libraries.

Node.js - The most mature and widely used runtime environment, offering a rich ecosystem of NPM libraries and stability in production. It stands out for its excellent I/O support and support for long-term projects.

Deno - A modern runtime environment created by Ryan Dahl, the creator of Node.js, with a strong focus on security and compliance with modern ECMAScript standards. Deno integrates TypeScript directly, eliminating the need for additional configuration.

Bun - The latest runtime environment, focused on performance and ease of use. It features fast start-up times and low resource consumption, making it a promising choice for high-performance applications.

The analyses conducted indicate that each environment has its own unique advantages and disadvantages that make them suitable for different types of projects:

\bigskip

\textbf{Keywords}: JavaScript, Performance, Deno, Bun, NodeJS
