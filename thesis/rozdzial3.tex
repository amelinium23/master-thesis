\section{Eksperymenty}
W tym rozdziale opisane zostaną eksperymenty przeprowadzone w celu zbadania testów wydajnościowych środowisk uruchomieniowych. Testy zostały przeprowadzone na jednym komputerze wyposażonym w system operacyjny Linux, co pozwoliło na zminimalizowanie wpływu innych czynników na wyniki testów. 

\subsection{Algorytmy sortowania}
W celu zbadania wydajności danego środowiska uruchomieniowego, skonstruowana odpowiednie eksperymenty, które sprawdzają wydajność algorytmu sortowania. Wszystkie algorytmu sortowania zostały przetestowane dla każdego środowiska. W tabeli \ref{tab:sorting_experiments} przedstawiono ilość iteracji oraz ilość eksperymentów dla przeprowadzonych eksperymenty.

\begin{table}[H]
  \centering
  \begin{tabular}{|c|c|}
    \hline
    \textbf{Liczba iteracji} & \textbf{Liczba elementów} \\ \hline
    10 & 1000 \\ \hline
    100 & 1000 \\ \hline
    1000 & 1000 \\ \hline
    10 & 10000 \\ \hline
    100 & 10000 \\ \hline
    1000 & 10000 \\ \hline
  \end{tabular}
  \caption{Parametry eksperymentów}
  \label{tab:sorting_experiments}
\end{table}

\subsubsection{Wyniki - sortowanie bąbelkowe}
\begin{figure}[H]
  \minipage{0.49\textwidth}
    \includegraphics[width=\linewidth]{Figures/sorting/bubble/e1_js.png}
    \caption{Wyniki eksperymentów dla algorytmu sortowania bąbelkowego dla 10 iteracji i 1000 elementów}
    \label{fig:bubble_sorting_e1}
  \endminipage\hfill
  \minipage{0.49\textwidth}
  \includegraphics[width=\linewidth]{Figures/sorting/bubble/e1_memory_js.png}
  \caption{Wyniki eksperymentów dla algorytmu sortowania bąbelkowego dla 10 iteracji i 1000 elementów}
  \label{fig:bubble_sorting_e1_memory_js}
  \endminipage\hfill
\end{figure}

\subsubsection{Wyniki - sortowanie szybkie}

\subsubsection{Wyniki - sortowanie pozycyjne}

\subsection{Algorytmy kodowania}
W celu zbadania wydajności możliwości kodowania dla środowisk, użyto algorytmu kodowania Base64, który jest najpopularniejszym algorytmem kodowania wykorzystywanym w aplikacjach webowych. W tabeli \ref{tab:encoding_experiments} przedstawiono liczbę przeprowadzonych eksperymentów, długość kodowanego słowa.

\begin{table}[H]
  \centering
  \caption{Parametry eksperymentów algorytmu kodowania Base64}
  \begin{tabular}{|c|c|}
    \hline
    \textbf{Liczba eksperymentów} & \textbf{Długość słowa}\\ \hline
    10 & 8192 \\ \hline
    100 & 32768 \\ \hline
  \end{tabular}
  \label{tab:encoding_experiments}
\end{table}

\subsubsection{Wyniki}

\subsection{Testy wydajnościowe operacji zapisu i odczytu plików}
W celu zbadania wydajności operacji zapisu oraz odczytu plików, przeprowadzono eksperymenty, które polegały na zapisie oraz odczycie plików o różnych rozmiarach. W przeprowadzonych testach wykorzystano pliki tekstowe. Do wypełnienia plików przykładowym tekstem. Zawartość plików została wygenerowana za pomocą modułu \textit{faker.js}. W tabeli \ref{tab:file_experiments} przedstawiono liczbę przeprowadzonych eksperymentów oraz rozmiar pliku.

\begin{table}[H]
  \centering
  \caption{Parametry eksperymentów zapisu i odczytu plików}
  \begin{tabular}{|c|c|}
    \hline
    \textbf{Liczba eksperymentów} & \textbf{Rozmiar pliku}\\ \hline
    10 & 1MB \\ \hline
    100 & 10MB \\ \hline
    1000 & 100MB \\ \hline
  \end{tabular}
  \label{tab:file_experiments}
\end{table}

\subsubsection{Wyniki}

\subsection{Testy wydajnościowe serwera HTTP}
W celu zbadania wydajności serwera HTTP, przeprowadzono testy obciążeniowe. Do testów została wykorzystany program \textit{oha} \cite{oha}, który pozwala na połączenie się do serwera HTTP oraz wysyłanie żądań z kilku połączeń jednocześnie. W tabeli \ref{tab:http_experiments} przedstawiono liczbę przeprowadzonych eksperymentów oraz ilość połączeń.

\begin{table}[H]
  \centering
  \caption{Parametry eksperymentów serwera HTTP}
  \begin{tabular}{|c|c|}
    \hline
    \textbf{Liczba eksperymentów} & \textbf{Ilość połączeń}\\ \hline
    10 & 100 \\ \hline
    100 & 1000 \\ \hline
    1000 & 10000 \\ \hline
  \end{tabular}
  \label{tab:http_experiments}
\end{table}

\subsubsection{Wyniki}

\subsection{Testy wydajnościowe zapisu i odczytu danych z bazy danych}
W celu zbadania wydajności operacji zapisu oraz odczytu danych z bazy danych, przeprowadzono testy wydajnościowe z wykorzystaniem bazy danych SQLite. W celu wygenerowania danych do bazy danych, wykorzystano moduł \textit{faker.js}. W tabeli \ref{tab:database_experiments} przedstawiono liczbę przeprowadzonych eksperymentów oraz ilość rekordów.

\begin{table}
  \centering
  \caption{Parametry eksperymentów zapisu i odczytu danych z bazy danych}
  \begin{tabular}{|c|c|}
    \hline
    \textbf{Liczba eksperymentów} & \textbf{Ilość rekordów}\\ \hline
    10 & 1000 \\ \hline
    100 & 10000 \\ \hline
    1000 & 100000 \\ \hline
  \end{tabular}
  \label{tab:database_experiments}
\end{table}

\subsubsection{Wyniki}
