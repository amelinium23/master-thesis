\section*{Metodologia badań}
\addcontentsline{toc}{section}{Metodologia badań}

W tym rozdziale przedstawiono metodologię przeprowadzonych badań, środowisko testowe oraz wybrane algorytmy do badań środowisk uruchomieniowych.
\subsection*{Środowisko testowe}
\addcontentsline{toc}{subsection}{Środowisko testowe}
Do przeprowadzenia testów został wykorzystany system operacyjny Linux Ubuntu 22.04.03, zainstalowany w (ang. \textit{ang. Windows Subsystem for Linux}). Specyfikacje sprzętowe komputera, na którym przeprowadzono testy, przedstawiono w tabeli \ref{tab:specyfikacje_sprzetowe}.
\begin{table}[H]
    \centering
    \begin{tabular}{|c|c|}
        \hline
        \textbf{Parametr} & \textbf{Wartość} \\
        \hline
        Procesor & Intel Core i7-13700HX \\
        \hline
        Pamięć RAM & 32 GB \\
        \hline
    \end{tabular}
    \caption{Specyfikacje sprzętowe komputera}
    \label{tab:specyfikacje_sprzetowe}
\end{table}

Wybór powyższego systemu jest podyktowany faktem, że jest to system, na którym wybrane środowiska są najczęściej uruchamiane. Badanie wykonane w 2015 roku, pokazuje iż większość osób odpowiedzialnych za administrowanie aplikacjami webowymi korzysta z systemu Linux \cite{performance_comparison_linux}. Kolejnym badaniem przeprowadzonym w 2021 roku, pokazało iż sam system jest bardziej wydajny niż Windows Server \cite{web_server_performance}. Badania także pokazało, że większość serwerów internetowych działa na systemie Linux. W związku z tym, wybór systemu Linux jest uzasadniony.

\subsection*{Wybrane algorytmy}
\addcontentsline{toc}{subsection}{Wybrane algorytmy}