\section*{Wstęp}
\addcontentsline{toc}{section}{Wstęp}

\subsection*{Problematyka}
\addcontentsline{toc}{subsection}{Problematyka}
\markboth{Problematyka}{Problematyka}
W czasach, których żyjemy jesteśmy przyzwyczajeni do niemalże natychmiastowego dostępu do informacji. Aby utrzymać ten dostęp do informacji należy to od serwerów, które mają dostęp do tych informacji. Wymienione serwery, często korzystają z różnych środowisk uruchomieniowych.

Specjalnie dla sieci \nomrefpage{WWW}, został stworzony język JavaScript, który miał być wykorzystywany w przeglądarkach. Wydajność samego języka na samym początku pozostawiała dużo do życzenia, jednakże obecnie jest zdecydowanie szybszy. Obecnie język stał się językiem ogólnego użytku, jest używany nie tylko do tworzenia interakcji z przeglądarką, a także jako serwer wysyłający zapytania \nomrefpage{HTTP} do użytkowników.

Na przestrzeni lat jedynym środowiskiem, które posiadało bardzo duże powodzenie jest NodeJS. Tą zależność pokazuje coroczna ankieta State Of Js \cite{State_of_js:2021} \cite{State_of_js:2022}, która zbiera wyniki od deweloperów. Deweloperzy odpowiadają czy znają daną technologię, czy używali jej w przeszłości. W tej ankiecie możemy zauważyć, że na przestrzeni lat pojawili się konkurenci dla NodeJS, tymi alternatywami są Deno i Bun. Jeżeli porównamy wyniki z 2022 roku oraz 2021, możemy zauważyć, że Deno wzmocniło swoją pozycje na rynku. Deweloperzy zaczęli coraz częściej korzystać z nowej technologi. W State of JS z wymienionych powyżej lat nie możemy jednak otrzymać informacji o środowisku nazwanym Bun jest to spowodowane wydaniem wersji 1.0 we wrześniu 2023.

Wymieniona powyżej ankieta mówi tylko o znajomości oraz popularności danego środowiska, aby dowiedzieć się jak dokładnie wygląda praca w danym środowisku musimy wykonać szereg działań, aby przekonać się o jego wydajności. Powstała duża liczba artykułów, które poruszały temat wydajności danego środowiska, jednak nie jesteśmy w stanie stwierdzić w jakie testy przeprowadził twórca artykułu. Dodatkowo nie są one udostępnione dla czytelnika artykułu.